%!TEX root = ../template.tex
%%%%%%%%%%%%%%%%%%%%%%%%%%%%%%%%%%%%%%%%%%%%%%%%%%%%%%%%%%%%%%%%%%%%
%% Config/7_package.tex
%% NOVA thesis configuration file
%%%%%%%%%%%%%%%%%%%%%%%%%%%%%%%%%%%%%%%%%%%%%%%%%%%%%%%%%%%%%%%%%%%%

\typeout{NT FILE Config/5_packages.tex}%

% \ntlangsetup{pt/contentsname=O MEU ÍNDICE}
% \ntlangsetup{en/contentsname=MY LITTLE TOC}




%%============================================================
%%  ADD YOUR PACKAGES AND OTHER CUSTOMIZATIONS HERE
%%============================================================

% \usepackage{widows-and-orphans} % add warnings about widows and orphans to “template.log”
% \usepackage{refcheck}     % list used and unused labels

\usepackage{paralist}     % To enable costumizble enumerates.  See the documentation.

\usepackage[textsize=footnotesize]{todonotes}  % To register TODO notes in the text
\usepackage{ifthen,xkeyval,xcolor,tikz,calc,graphicx}

\usepackage{makecell}      % Multilie table cells and control vertical and horizontal alignment

\usepackage{indentfirst}       % Indent first list after a (sub)section title

\usepackage{physics}
\usepackage{xfrac}
\usepackage{multirow}
\usepackage{subcaption}
\usepackage{pgfgantt}

\usepackage[normalem]{ulem}

\usepackage{tikz}
\tikzstyle{startstop} = [rectangle, rounded corners, minimum width=3cm, minimum height=1cm,text centered, draw=black]
\tikzstyle{io} = [trapezium, trapezium left angle=70, trapezium right angle=110, minimum width=3cm, minimum height=1cm, text centered, draw=black]
\tikzstyle{process} = [rectangle, minimum width=3cm, minimum height=1cm, text centered, draw=black]
\tikzstyle{decision} = [diamond, minimum width=3cm, minimum height=1cm, text centered, draw=black]
\tikzstyle{arrowdiag} = [thick,->,>=stealth]
\usetikzlibrary{snakes,arrows.meta}
\usetikzlibrary{shapes.multipart,shapes.geometric}


\usepackage{tikz-feynman}
\usepackage{listings}

\usepackage{xcolor}

\definecolor{codegreen}{rgb}{0,0.6,0}
\definecolor{codegray}{rgb}{0.5,0.5,0.5}
\definecolor{codepurple}{rgb}{0.58,0,0.82}
\definecolor{backcolour}{rgb}{0.95,0.95,0.92}

\lstdefinestyle{mystyle}{
    backgroundcolor=\color{backcolour},   
    commentstyle=\color{codegreen},
    keywordstyle=\color{magenta},
    numberstyle=\tiny\color{codegray},
    stringstyle=\color{codepurple},
    basicstyle=\ttfamily\footnotesize,
    breakatwhitespace=false,         
    breaklines=true,                 
    captionpos=b,                    
    keepspaces=true,                 
    numbers=left,                    
    numbersep=5pt,                  
    showspaces=false,                
    showstringspaces=false,
    showtabs=false,                  
    tabsize=2
}

\lstset{style=mystyle}