%!TEX root = ../template.tex
%%%%%%%%%%%%%%%%%%%%%%%%%%%%%%%%%%%%%%%%%%%%%%%%%%%%%%%%%%%%%%%%%%%
%% chapter4.tex
%% NOVA thesis document file
%%
%% Chapter with introduction
%%%%%%%%%%%%%%%%%%%%%%%%%%%%%%%%%%%%%%%%%%%%%%%%%%%%%%%%%%%%%%%%%%%

\typeout{NT FILE chapterNone.tex}%

\chapter{A new High-Performance Computing code for the parallelization of atomic structure calculations}
\section{A brief introduction on \gls{MPI}}


\section{Program advantages}
\section{Program limitations}
Blocking communication

CPU overheat

No buffering for now
\chapter{Fundamental atomic parameters calculation}
asdas
\chapter{Spectra simulation}
asdas

\chapter{Spectra analysis}
\section{Photoexcitation cross-section estimation}
\section{Photoionization cross-section computation}

\chapter{Comparison with experimental data}
\chapter{Next Steps}
\chapter{Final remarks and Conclusion}
---------------------------------------------------

\todo{Usar o Denitions aqui}

\section{Quantum states and properties}

\section{State transitions}

\subsection{Transition rates and widths}
\subsection{Branching Ratios and Fluorescence Yields}

\section{Spectra simulation}

\chapter{Spectra Analysis}


Explicar a lorentziana assimetrica. Mostrar que se pode fazer assim e estudar a derivada a x=media

Both branches and derivatives are continuos

\chapter{Development of a \gls{HPC} script for the parallelization of \gls{ASC}}

\section{blablabla}

\section{Speedup}
\todo{Compare with Amdahl's law or Gustafson's law. }
\todo{Comparação com o do Jorge}

