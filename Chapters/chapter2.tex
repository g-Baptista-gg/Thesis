%!TEX root = ../template.tex
%%%%%%%%%%%%%%%%%%%%%%%%%%%%%%%%%%%%%%%%%%%%%%%%%%%%%%%%%%%%%%%%%%%
%% chapter2.tex
%% NOVA thesis document file
%%
%% Chapter with introduction
%%%%%%%%%%%%%%%%%%%%%%%%%%%%%%%%%%%%%%%%%%%%%%%%%%%%%%%%%%%%%%%%%%%

\typeout{NT FILE chapter2.tex}%


\chapter{Atomic Structure Calculations}\label{cha:atom_calc}
When studying a system composed of multiple charged bodies, one must consider all the existing interactions. Whilst there are known analytical solutions for the 2-bodies Hydrogenoid systems, with the presence of more non spatially-bound particles, these become impossible to find for Schrödinger's equation, even using Born-Oppenheimer's approximation, where the nucleus is considered static, at a fixed position, making the electrons the only non spatially-bound particles in the system. That way, there was a need for the development of numerical solutions to solve this problem.

\section{The non-relativistic Hamiltonian}

The first approach used in order to solve the many-bodies problem used a non-relativistic consideration. This way, the Hamiltonian consisted on the sum of the system's non-relativistic momentum-related energies and the Coulomb interactions between bodies, while still considering the Born-Oppenheimer approximation.

Essentially, and in atomic units:




\begin{equation}
    \begin{gathered}
        \underbrace{\sum_i^N \overbrace{ \dfrac{1}{2}\laplacian_i}^{E_1}  \overbrace{-\frac{Z}{r_i}}^{E_2}}_{\text{Individual Hamiltonian}}  + \underbrace{\sum_{i<j}^{j}\overbrace{\dfrac{1}{r_{ij}}}^{E_3}}_{\text{Pair repulsion}}\\
        E_1\rightarrow\text{Momentum}\quad E_2\rightarrow\text{e$^-$nuc. Coulomb attraction}\quad E_3\rightarrow\text{e$^-$e$^-$ Coulomb repulsion}
    \end{gathered}
    \label{eq:nonRelHam}
\end{equation}



\subsection{The Hartree-Fock Method}\label{sec:HF}
\textit{Note: This section used the works in~\cite{HFweb,JPS,Ramon,Blinder2018} as reference.}

Hartree developed an iterative method, further enhanced by Fock and Slater, based on the field self-consistency method.

In this method, while studying a multi-electronic system, such as an atom, each electron's wavefuction is composed as a product of a spacial part, $\psi$, and one indicating the electron's spin, $\chi$, as to be able to account for relativistic effects.

\begin{equation}
    u=\psi\chi
\end{equation}

The wavefuction capable of describing the whole system, $\Psi$, should be somewhat of a product of all the wavefunctions describing each individual electron. However, one must not forget the need for this wavefunction to respect the antisymmetry principle, due to the electron's fermionic nature. In order to respect this, $\Psi$ is to be composed of a Slater determinant:

\begin{equation}
    \Psi=\frac{1}{\sqrt{N!}}\mqty| u_1(x_1) & u_2(x_1) & \cdots & u_N(x_1)\\
    u_1(x_2) & u_2(x_2) & \cdots & u_N(x_2)\\
    \vdots & \vdots& \ddots&\vdots\\
    u_1(x_N)&u_2(x_N)&\cdots&u_N(x_N)&|
\end{equation}

It is of high importance that the wavefunctions must form an orthonormal basis. These are to be initialized as trial wavefunctions. 


The main goal of this method is to follow the variational principle and to minimize a functional, with the purpose of minimizing the system's energy. This optimal, yet unknown energy, $E_0$ (calculated by Operating the Hamiltonian on the optimal wavefunctions), will try to be reached by variating the trial functions that provide a non-minimized solution, with $\ev**{H}{\Psi}\geq E_0$.

The computational method consists on starting with the previously mentioned trial wavefunctions, using them to calculate Hartree-Fock's potential through the HF equations.% \textbf{Acho que aqui nao tenho espaço de as explicar. ou meto em appendix ou falo so ja na tese} 


In a very simplified way, the self-consistent Hartree-Fock computational method can be represented by the block diagram in Figure~\ref{fig:diagram}.


\begin{figure}[h!]
    \centering
    \begin{tikzpicture}[node distance=2cm,every text node part/.style={align=center}]
        \node (initWF) [startstop] {Define Initial \\ Test Wavefunctions};
        \node (loadWF) [io, below of=initWF, yshift=-0cm] {Initial \\ Wavefunctions};
        \node (EnCalc1) [process,right of = loadWF,xshift= 2cm]{Compute \\ Energy ($E_i$)};
        \node (calcPotential) [process,below of= loadWF]{Compute density matrix \\ and HF potential};
        \node (calcWF) [process, below of = calcPotential]{Solve HF equations \\ for new potential};
        \node (newWF) [io, below of=calcWF] {New \\ Wavefunctions};
        \node (EnCalc2) [process,right of = newWF,xshift= 2cm]{Compute \\ Energy ($E_f$)};
        \node (EnComp) [decision,below of = newWF,yshift=-1cm]{$E_i-E_f\leq \lambda$?};
        \node (stop) [startstop,right of = EnComp,xshift=2cm] {End of \\ algorithm};

        \draw [arrowdiag] (initWF) -- (loadWF);
        \draw [arrowdiag] (loadWF) -- (EnCalc1);
        \draw [arrowdiag] (loadWF) -- (calcPotential);
        \draw [arrowdiag] (calcPotential) -- (calcWF);
        \draw [arrowdiag] (calcWF) -- (newWF);
        \draw [arrowdiag] (newWF) -- (EnCalc2);
        \draw [arrowdiag] (newWF) -- (EnComp);
        \node (aux1) [left of =EnComp,xshift=-2cm] {No};
        \node (aux2) [left of =loadWF,xshift=-2cm] {Load new WF \\ as initial WF};
        \draw (EnComp) -- (aux1);
        \draw (aux1) -- (aux2);
        \draw [arrowdiag] (aux2) -- (loadWF);
        \draw [arrowdiag] (EnComp) --node[anchor=south]{Yes} (stop);
    \end{tikzpicture}
    \caption{HF method's block diagram.}\label{fig:diagram}
\end{figure}





\section{The Dirac Equation}

\textit{Note: This section used the works in~\cite{Thaller1992,Beyer2016,Sakurai2020,Bethe1977} as reference.}


It is no secret that the Schrödinger equation has some very considerable limitations. The fact that it does not account for the existence of the electron's spin and the lack of consideration of relativistic effects are some of the most impactful setbacks.
%------------------------------------------------------------------------------




Many scientists, such as Klein, Gordon and later Fock, had already conceived a relativistic correction to Schrödinger's equation, where the free-particle energy makes use of the relativistic momentum-energy relation \eqref{eq:EnMomRel}.


 \begin{equation}
    E=\sqrt{c^2\vb*{p}^2+m^2c^4}
    \label{eq:EnMomRel}
 \end{equation}


 Which can be derived from the \gls{lorentzinvariance} \gls{MinkNorm} of the momentum \gls{fourvec}~\eqref{eq:4mom}.


 \begin{equation}
    p^\mu p_\mu = m^2c^2\Leftrightarrow\dfrac{E^2}{c^2}-\vb*{p}^2=m^2c^2 \Leftrightarrow  \dfrac{E^2}{c^2}= \vb*{p}^2 +m^2c^2
    \label{eq:4mom}
\end{equation}

 Now, inputting this new energy operator into Schrödinger's equation, yields the Klein-Gordon equation \eqref{eq:KG}, allowing for Schrödinger's equation now to be Lorentz-invariant.

 \begin{equation}
    -\hbar^2 \pdv[2]{t} \vb*{\psi}=\qty(-c^2\hbar^2\laplacian + m^2c^4)\vb*{\psi}
    \label{eq:KG}
\end{equation}



 This new approach was, however, still faulty, due to only describing spin 0 particles (e.g., some mesons), and making use of a second order derivative in the time-like component.

 That way, a new equation was developed by Paul Dirac, in 1928~\cite{Dirac}, one taking into account not the classical 3 dimensional space components, but the relativistic four components.




 Dirac took a spin at rewriting the energy-momentum relation, ending up with an equivalent equation\eqref{eq:diracEnergy}, involving $4\times 4$ matrices, due to the 4 relativistic dimensions at play, and incorporating spins into the equation by making use of the now famous Pauli matrices~\eqref{eq:pauli}.

 \begin{equation}
    E=c\vb*{\alpha}\cdot \vb*{p}+{\beta} mc^2,\quad \vb*{\alpha}=\qty(\alpha_1,\alpha_2,\alpha_3)
    \label{eq:diracEnergy}
 \end{equation}

 \begin{align}
    \alpha_i&=\mqty(0 & {\sigma}_i\\ {\sigma}_i&0) & {\beta}&=\mqty(I_2 & 0 \\ 0 &-I_2) &I_2&=\mqty(1&0\\0&1)\\
    \sigma_1&=\mqty(0&1\\1&0) & \sigma_2&=\mqty(0&-i\\i&0)& \sigma_3&=\mqty(1&0\\0&-1)\label{eq:pauli}
 \end{align}

 In order to fully comprehend this shift of notation, one should equate the square of the two equations, \eqref{eq:EnMomRel} and \eqref{eq:diracEnergy}, and confirm if logic still stands.

 \begin{equation}
    c^2 \vb*{p}^2 + m^2c^4 = c^2\vb*{\alpha}^2\vb*{p}^2+2 m c^3 \vb*{\alpha}\cdot \vb*{p} \cdot \beta + \beta^2 m^2 c^4
 \end{equation}

 In order for this equation to make sense, the following conditions must be true (which in fact, they are):

\begin{gather}
    \begin{cases}
        c^2\vb*{p}^2=c^2\vb*{\alpha}^2 p^2  &\Leftrightarrow \vb*{\alpha}^2=1\\
        0=2mc^3  p \vb*{\alpha} \beta &\Leftrightarrow \vb*{\alpha} \beta =0\\
        m^2c^4=\beta^2 m^2c^4     &\Leftrightarrow \beta^2 =1
    \end{cases}
\end{gather}


With all the previous considerations taken into account, one can now construct Dirac's free-particle equation\eqref{eq:DiracEq}:

\begin{equation}
    i\hbar \pdv{t}\vb*{\psi}=\qty( c\vb*{\alpha}\cdot \vb*{p} +\beta m c^2)\vb*{\psi}=\mqty(mc^2 I_2& -i\hbar c \vb*{\sigma}\cdot \grad\\-i\hbar c \vb*{\sigma}\cdot \grad & -mc^2 I_2)\cdot \mqty(\psi_1\\\vdots\\\psi_4)
    \label{eq:DiracEq}
\end{equation}

This equation, however, as mentioned above, can only describe a single particle present in a field-free region. In order to account for the existence of a field, such as the electromagnetic field, derived from the four-potential $A^\mu$, composed by the electric scalar potential field, $A^0=\phi$, and the vector potential, $(A^1,A^2,A^3)=\vb*{A}$, the following change to the momentum four-vector must be made:


\begin{equation}
    p^\mu \rightarrow p^\mu - e A^\mu,\quad A^\mu = \qty(\phi,\vb*{A})
\end{equation}

The Hamiltonian can now be rewritten to account for the presence of the electromagnetic field~\eqref{eq:DiracEqField}. This way it is possible to include, for example, the electron-nucleus Coulomb attraction.

\begin{equation}
    H_D=-e\phi + \beta m c^2 + \vb*{\alpha}\qty(c\vb*{p}+e\vb*{A})
    \label{eq:DiracEqField}
\end{equation}

For a central potential, as is the one generated by the nuclear charge (assuming Born-Oppenheimer's approximation), the 3 space-like components from the four-potential are null, and the time-like component, $\phi=\dfrac{Z e}{r}$. The Hamiltonian gains now a more recognizable form:

\begin{equation}
    H_D=-\frac{e^2 Z}{r} + \beta m c^2 + \vb*{\alpha}\cdot\vb*{p}c
    \label{eq:diracHam}
\end{equation}

Something very interesting about Dirac's equation is the fact that it does not yield a single solution, but in fact, two: the large component (positive energy values), for particles, and the small component (negative energy values), for antiparticles.

\section{The Dirac-Breit Equation}

Once again, when considering a system composed of many bodies, one must consider all the present interactions, namely, the electron-electron repulsion in an atom.
 Breit, in 1929, had created a relativistic approach to treat the electron-electron interactions, consisting on a set of equations building upon the classical non-relativistic Hamiltonian from equation \eqref{eq:nonRelHam}, which can be consulted in Appendix \ref{ap:Breit}.
  Breit's equations are able to account for angular momenta couplings and estimate level energy splittings, the change of a particle's apparent mass as a function of velocity, and even include the interaction of an applied external magnetic field~\cite{Bethe1977}.

  It is quite obvious Breit's equations introduce a great complexity in the search of the new Hamiltonian's eigenfunctions. However, when trying to include an approximation of Breit's considerations into Dirac's equation, one must add the following operator to the one present in equation \eqref{eq:diracHam}:

  \begin{equation}
    H_B = \sum_{i>j}\frac{e^2}{r_{ij}} - e^2\qty(\frac{\vb*{\alpha}_i\vb*{\alpha}_j}{r_{ij}}+\frac{\qty(\vb*{\alpha}_i \grad_i)\qty(\vb*{\alpha}_j \grad_j)r_{ij}}{2})  
    \label{eq:BreitHam}
  \end{equation}

This set of terms will account for the fact that Coulomb interactions, mediated by the electric field, and therefore, \gls{virtual photons}, cannot occur at instantaneous velocities, but at the speed of light.

\section{The \gls{MCDF} Method}

As previously mentioned in section \ref{sec:HF}, there is a need for a numerical method in order to compute and find the eigenfunctions for a many-body Hamiltonian. While the Hartree-Fock method was able to reasonably solve the non-relativistic problem, now, while considering the Dirac-Breit Hamiltonian from equations \eqref{eq:diracHam} and \eqref{eq:BreitHam}, there is a need for a new method.

Hence, the state of the art \gls{MCDFGME} arises. This self-consistent iterative method, based on the same method present in section \ref{sec:HF}, is able to solve and find eigenfunctions for a multielectronic system, now taking into account the Dirac-Breit Hamiltonian. Moreover, it is also capable of incorporating electron correlation and many QED effects not yet considered in the relativistic equation, such as the Lamb-shift, vacuum polarization, and the electron's self energy.
A brief description of these contributions can be found in  appendix \ref{ap:QED}


