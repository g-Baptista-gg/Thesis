\typeout{NT FILE chapter6.tex}%

\chapter{Conclusions and Final Remarks}

In this thesis, atomic structures calculations were performed for many of Copper's excited states. This involved complex computations for the thousands of energy levels generated by different possible quantum number configurations. For the purpose of performing the computation in a sustainable manner, a parallelized calculation deployment tool was developed and implemented.

For the calculated levels, radiative decay and Auger transitions were the two decay processes considered for the initial atomic state. Calculations were also performed for the radiative transitions that occur for post auger transitions states. With the obtained transition probability values, the Fluorescence Yield for each subshell was calculated for the different excitations.

With the absence of a calculation method for photoexcitation cross-sections, a new consideration was formulated as to be able to compute a scaling factor for spectra simulation based on rate normalization. A conversion process was also implemented for photoionization oscillator strengths as to make them compatible with the previously mentioned scaling factors.

Synthetic spectra were simulated for the $K{_\alpha}$ transitions in a system exposed to an x-ray beam with energies surrounding the ionization threshold. The line profiles were then studied and fitted with asymmetric Lorentzian profiles as a function of the energy of incident radiation, with the influences of excitations of an $1s$ electron to orbitals $4s$, $4p$ and $5p$ being noticed in the evolution of the profile parameters.

The obtained results were compared with a recent high-precision spectrum obtained while measuring Copper x-ray transitions when exposed to tunable radiation from a synchrotron line and measured with a high-resolution Double Crystal Spectrometer.

Further studies should be performed on this subject. For a more accurate theoretical spectrum, R-Matrix calculations should be performed for computing the proper cross-sections of the relevant physical processes and low-energy transitions between different excited states should be accounted for. The new improved results should then be compared with the experimental result from a high-precision x-ray fluorescence study for vaporized Copper, as to mitigate the solid state effects observed in the spectrum used in this study.

