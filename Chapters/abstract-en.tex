%!TEX root = ../template.tex
%%%%%%%%%%%%%%%%%%%%%%%%%%%%%%%%%%%%%%%%%%%%%%%%%%%%%%%%%%%%%%%%%%%%
%% abstrac-en.tex
%% NOVA thesis document file
%%
%% Abstract in English([^%]*)
%%%%%%%%%%%%%%%%%%%%%%%%%%%%%%%%%%%%%%%%%%%%%%%%%%%%%%%%%%%%%%%%%%%%

\typeout{NT FILE abstrac-en.tex}%

The work performed on this thesis comes as part of the effort to further understand the highly convoluted structure present on Copper's X-ray emission spectra, where, as with many other transition metals, a skewness can be observed on the K$_{\alpha_{1,2}}$, K$_{\beta}$ and L transition lines. These lines originate due to the radiative relaxation of the atom's electronic structure post-ionization of inner shell electrons.
However, it is very likely that the observed skewness is due to copper's satellite states' transitions.

Throughout this thesis, a study will be performed for the satellite states formed by the excitation of the inner-shell electrons, where, as opposed to the ionization process, usually considered in X-ray calculations, a photoexcitation process occurs.

 Multiple atomic structure calculations will be performed using the \textit{ab initio} state of the art \gls{MCDFGME} code for different excited states configurations.

 The obtained results will then be used in the analysis of experimental data obtained from a High-Precision \gls{DCS}, using a synchrotron X-ray source.

Due to the complexity of the calculations, the process can become substantial in terms of computational power and time. Therefore, further similar and more complex studies will be performed by implementing and running a script in the \textit{Oblivion} supercomputer located at the University of Évora.

\keywords{
  Atomic Excitation \and
  X-ray lines \and
  \gls{MCDFGME}\and
  \gls{DCS}\and
  High Performance Computing
}
