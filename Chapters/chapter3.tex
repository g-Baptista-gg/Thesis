%!TEX root = ../template.tex
%%%%%%%%%%%%%%%%%%%%%%%%%%%%%%%%%%%%%%%%%%%%%%%%%%%%%%%%%%%%%%%%%%%
%% chapter3.tex
%% NOVA thesis document file
%%
%% Chapter with introduction
%%%%%%%%%%%%%%%%%%%%%%%%%%%%%%%%%%%%%%%%%%%%%%%%%%%%%%%%%%%%%%%%%%%

\typeout{NT FILE chapter3.tex}%

\chapter{Calculation of fundamental parameters}

With every transition's energy and rate calculated, certain parameters of great relevance for spectra simulation, or even by their own merit can be now be computed. 

\section{Fluorescence Yield}

This parameter is, by itself, one of the most impactful\todo{alguma cena aqui}. The Fluorescence Yield value for a given 1-hole state reflects the probability of said state to decay via radiative decay, as opposed to auger emission. In order to compute it, a few steps must be taken:

First, a reminder that, for each computed level $i$, due to its degeneracy, there is actually an amount of $g_i$ states sharing the same properties:

\begin{equation}
    g_i\equiv g\qty[(n,l_j)_i,J_i,\epsilon_i]=2J_i+1,
    \label{eq:degen}
\end{equation}

That level's total radiative rate, $R_i^{R}$ can now be computed as the sum of the rates for every transition from that level to another 1-hole level, denoted by $f$:

\begin{equation}
    R_i^{R} = g_i \sum_f \cdot R_{i,f}^{R},
\end{equation}

The same procedure can be followed for the non-radiative rate, $R_i^{NR}$ considering the final 2-hole levels:

\begin{equation}
    R_i^{R} = g_i \sum_f \cdot R_{i,f}^{NR},
\end{equation}

That level's fluorescence yield, $\omega_i$ can now simply be calculated by:

\begin{equation}
    \omega_i=\frac{R_i^{R}}{R_i^{R}+R_i^{NR}},
\end{equation}

This quantity, however, may not be much relevant by itself. When discussing fluorescence Yields, usually we are interested in the subshell's yield. In order to calculate this value for the desired subshell, eg. K$_1$, the values for both rates should be summed for every level with a hole present in the subshell at study.

The values of the fluorescence yields for every calculated excited state can be found in Table~\ref{tab:FY}:

\begin{table}[h!]
    \centering
    \caption{Fluorescence Yield for the first 9 subshells.}
    \label{tab:FY}
    \rowcolors{1}{}{GhostWhite}
    \begin{tabular}{c||c|c|c|c|c|c|c|c|c}
        \toprule Atomic&\multicolumn{9}{c}{Fluorescence Yield}\\
        System&K$_1$&L$_1$&L$_2$&L$_3$&M$_1$&M$_2$&M$_3$&M$_4$&M$_5$\\
        \midrule
        Cu$^{1+}$ \\ 
        4s&  \\
        4p &  &\\
        4d & &\\
        4f & \\
        5s & \\
        5p & \\
        5d & \\
        5f & \\
        5g & \\
        6s & \\
        6p & \\
        6d & \\
        6f & \\
        6g & \\
        6h\\
        \bottomrule
    \end{tabular}
\end{table}